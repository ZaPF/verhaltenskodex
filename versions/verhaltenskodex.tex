% Options for packages loaded elsewhere
\PassOptionsToPackage{unicode,pdfcreator=\{pdflatex\},pdfdisplaydoctitle=true}{hyperref}
\PassOptionsToPackage{hyphens}{url}
%
\documentclass[
  a4paper,
  oneside]{scrartcl}
\usepackage{amsmath,amssymb}
\usepackage{iftex}
\ifPDFTeX
  \usepackage[T1]{fontenc}
  \usepackage[utf8]{inputenc}
  \usepackage{textcomp} % provide euro and other symbols
\else % if luatex or xetex
  \usepackage{unicode-math} % this also loads fontspec
  \defaultfontfeatures{Scale=MatchLowercase}
  \defaultfontfeatures[\rmfamily]{Ligatures=TeX,Scale=1}
\fi
\usepackage{lmodern}
\ifPDFTeX\else
  % xetex/luatex font selection
\fi
% Use upquote if available, for straight quotes in verbatim environments
\IfFileExists{upquote.sty}{\usepackage{upquote}}{}
\IfFileExists{microtype.sty}{% use microtype if available
  \usepackage[]{microtype}
  \UseMicrotypeSet[protrusion]{basicmath} % disable protrusion for tt fonts
}{}
\makeatletter
\@ifundefined{KOMAClassName}{% if non-KOMA class
  \IfFileExists{parskip.sty}{%
    \usepackage{parskip}
  }{% else
    \setlength{\parindent}{0pt}
    \setlength{\parskip}{6pt plus 2pt minus 1pt}}
}{% if KOMA class
  \KOMAoptions{parskip=half}}
\makeatother
\usepackage{xcolor}
\setlength{\emergencystretch}{3em} % prevent overfull lines
\providecommand{\tightlist}{%
  \setlength{\itemsep}{0pt}\setlength{\parskip}{0pt}}
\setcounter{secnumdepth}{5}
\ifLuaTeX
\usepackage[bidi=basic]{babel}
\else
\usepackage[bidi=default]{babel}
\fi
\babelprovide[main,import]{ngerman}
% get rid of language-specific shorthands (see #6817):
\let\LanguageShortHands\languageshorthands
\def\languageshorthands#1{}
\ifLuaTeX
  \usepackage{selnolig}  % disable illegal ligatures
\fi
\usepackage{bookmark}
\IfFileExists{xurl.sty}{\usepackage{xurl}}{} % add URL line breaks if available
\urlstyle{same}
\hypersetup{
  pdftitle={Verhaltenskodex der ZaPF},
  pdfauthor={Zusammenkunft aller Physikfachschaften},
  pdflang={de-DE},
  hidelinks,
  pdfcreator={LaTeX via pandoc}}

\title{Verhaltenskodex der ZaPF}
\author{Zusammenkunft aller Physikfachschaften}
\date{}

\begin{document}
\maketitle

\section{Vorbemerkung}\label{vorbemerkung}

\begin{quote}
Die ZaPF ist ein freies Forum von und für Physikstudika. Sie bietet eine
sichere Umgebung für Teilnehmika unabhängig ihrer Alter, Geschlechter,
sexueller Identitäten oder Orientierungen, physischen Erscheinungen und
Befähigungen, Studiengänge, Lebensumstände sowie politischer oder
religiöser Überzeugungen. Aus diesem Grund kann diskriminierendes,
ausschließendes und grenzüberschreitendes Verhalten in jeglicher Form
nicht toleriert werden.
\end{quote}

Stellungnahme der Zusammenkunft aller Physik-Fachschaften gegen
Diskriminierung, Ausschließung und grenzüberschreitendes Verhalten,
``beschlossen''* in Wien.

*Da das Abschlussplenumsprotokoll aus Wien verschwunden ist, kann der
Beschluss historisch nicht belegt werden.

\section{Verhaltenskodex}\label{verhaltenskodex}

Wir wollen die ZaPF gemeinsam in einem respektvollen Miteinander
gestalten. Wir wollen inklusive, diskriminierungsfreie und offene
Kommunikation miteinander. Beispielhaft als nicht abgeschlossene Liste:

\begin{itemize}
\tightlist
\item
  Wir freuen uns über neue Menschen auf der ZaPF und möchten eine
  Umgebung schaffen, in der wir voneinander lernen können.
\item
  Wir schätzen Meinungspluralismus und ermuntern Personen
  marginalisierter Gruppen, sich aktiv einzubringen.
\item
  Wir kommunizieren respektvoll miteinander, auch wenn wir verschiedener
  Meinung sind.
\item
  Wir bemühen uns bewusst darum, einander ausreden zu lassen und uns
  gegenseitig zuzuhören.
\item
  Wir gehen miteinander empathisch um.
\item
  Wir akzeptieren, wenn Menschen Fragen nicht beantworten wollen.
\item
  Wir haben körperlichen Kontakt nur mit expliziter Zustimmung;
  Nichtzustimmung wird ohne Begründung akzeptiert und nicht hinterfragt.

  \begin{itemize}
  \tightlist
  \item
    Wir respektieren, dass bereits gegebene Zustimmung jederzeit
    widerrufen werden kann.
  \item
    Wir nehmen Zustimmung grundsätzlich verbal an. Non-verbal kann eine
    Person nur zustimmen, wenn sie Kommunikation über non-verbalen
    Konsens von sich aus explizit und unmissverständlich anbietet, z.B.
    durch sichtbare Kennzeichnung auf dem Tagungsausweis.
  \end{itemize}
\item
  Wir schaffen Sicherheit und diskutieren in einer entspannten
  Atmosphäre miteinander.
\item
  Wir ermutigen rücksichtsvolles Fragen.
\item
  Wir wünschen uns couragiertes Verhalten.
\item
  Wir schützen unsere Grenzen und halten uns an die von anderen
  gesetzten Grenzen. Du selbst bestimmst, was für dich
  grenzüberschreitend ist.
\end{itemize}

Die ZaPFika engagieren sich, eine Atmosphäre von Sicherheit und
Gewaltlosigkeit für alle zu schaffen. Dies bedeutet insbesondere, dass
wir:

\begin{itemize}
\tightlist
\item
  keine Form von sexualisierter Gewalt, Belästigung oder Diskriminierung
  akzeptieren.
\item
  keine Beleidigungen, Trollen, Degradierungen oder persönliche Angriffe
  akzeptieren.
\item
  nicht akzeptieren, wenn Menschen anstelle von Argumenten angegriffen
  werden.
\end{itemize}

\section{Umsetzung}\label{umsetzung}

Das Umsetzen dieses Verhaltenskodex liegt in der Verantwortung aller
Anwesenden: Teilnehmika, Helfika, Orga, etc. Wie in allem
zwischenmenschlichem Handeln, kommt es jedoch auch auf der ZaPF zu
Problemen. Du kannst dich mit Problemen jederzeit an die
Vertrauenspersonen wenden. Im Konfliktfall mit anderen Menschen oder bei
persönlichen Problemen, die nur dich betreffen, kannst du ihnen von
deinem Problem erzählen. Falls du das wünschst, können sie mit anderen
Personen vermitteln, für dich bei der Tagungsleitung um Hilfe bitten
oder andere, z.B. externe, Hilfe organisieren. Du kannst mit so vielen
oder so wenigen Vertrauenspersonen sprechen, wie du möchtest.

\section{Geltungsbereich}\label{geltungsbereich}

Dieser Verhaltenskodex gilt auf allen unmittelbar mit der ZaPF in
Verbindung stehenden Veranstaltungen. Im Speziellen gilt er auf allen
ZaPFen und durch Organe der ZaPF organisierten Veranstaltungen für alle
anwesenden Personen. Außerdem für alle die ZaPF repräsentierenden
Personen während Sie ihre Funktion ausüben. Mit der Anmeldung zu einer
entsprechenden Veranstaltung, spätestens jedoch mit dem Erscheinen,
stimmst du zu, dich an diesen Verhaltenskodex zu halten.

\section{Änderungshistorie}\label{uxe4nderungshistorie}

Verabschiedet am 13. November 2022 auf der ZaPF in Hamburg.

Geändert auf der:

\begin{itemize}
\tightlist
\item
  Sommer-ZaPF 2023 in Berlin,
\item
  und auf der Sommer-ZaPf 2024 in Kiel.
\end{itemize}

\end{document}
