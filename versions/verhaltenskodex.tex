\documentclass[12pt,oneside]{scrartcl}
% generated by Docutils <https://docutils.sourceforge.io/>
\usepackage{cmap} % fix search and cut-and-paste in Acrobat
\usepackage{ifthen}
\usepackage[T1]{fontenc}
\setcounter{secnumdepth}{0}

%%% Custom LaTeX preamble


%%% User specified packages and stylesheets
% embedded stylesheet: preamble.tex
% Sprache und Encodings
\usepackage[ngerman]{babel}

% Typographisch interessante Pakete
\usepackage{microtype} % Randkorrektur und andere Anpassungen

% References to Internet and within the document !!!always last package!!!
\usepackage[pdftex,colorlinks=false,
pdftitle={Verhaltenskodex der ZaPF},
pdfauthor={Zusammenkunft aller Physikfachschaften},
pdfcreator={pdflatex},
pdfdisplaydoctitle=true]{hyperref}

% Absaetze nicht Einruecken
\setlength{\parindent}{0pt}

\renewcommand*\thesection{\arabic{section}}


%%% Fallback definitions for Docutils-specific commands

% hyperlinks:
\ifthenelse{\isundefined{\hypersetup}}{
  \usepackage[colorlinks=true,linkcolor=blue,urlcolor=blue]{hyperref}
  \usepackage{bookmark}
  \urlstyle{same} % normal text font (alternatives: tt, rm, sf)
}{}
\hypersetup{
  pdftitle={Verhaltenskodex der ZaPF},
}

%%% Body
\begin{document}
\title{Verhaltenskodex der ZaPF%
  \label{verhaltenskodex-der-zapf}}
\author{}
\date{}
\maketitle


\section{Vorbemerkung%
  \label{vorbemerkung}%
}

“Die ZaPF ist ein freies Forum von und für Physikstudika. Sie bietet eine
sichere Umgebung für Teilnehmika unabhängig ihrer Alter, Geschlechter, sexueller
Identitäten oder Orientierungen, physischen Erscheinungen und Befähigungen,
Studiengänge, Lebensumstände sowie politischer oder religiöser
Überzeugungen. Aus diesem Grund kann diskriminierendes, ausschließendes und
grenzüberschreitendes Verhalten in jeglicher Form nicht toleriert werden.”

Stellungnahme der Zusammenkunft aller Physik-Fachschaften gegen Diskriminierung,
Ausschließung und grenzüberschreitendes Verhalten, \textquotedbl{}beschlossen\textquotedbl{}* in Wien.

*Da das Abschlussplenumsprotokoll aus Wien verschwunden ist, kann der Beschluss
historisch nicht belegt werden.


\section{Verhaltenskodex%
  \label{verhaltenskodex}%
}

Wir wollen die ZaPF gemeinsam in einem respektvollen Miteinander gestalten. Wir
wollen inklusive, diskriminierungsfreie und offene Kommunikation
miteinander. Beispielhaft als nicht abgeschlossene Liste:

\begin{itemize}
\item Wir freuen uns über neue Menschen auf der ZaPF und möchten eine Umgebung
schaffen, in der wir voneinander lernen können.

\item Wir schätzen Meinungspluralismus und ermuntern Personen marginalisierter
Gruppen, sich aktiv einzubringen.

\item Wir kommunizieren respektvoll miteinander, auch wenn wir verschiedener Meinung
sind.

\item Wir lassen einander ausreden und hören uns zu.

\item Wir gehen miteinander empathisch um.

\item Wir akzeptieren wenn Menschen Fragen nicht beantworten wollen.

\item Wir haben körperlichen Kontakt nur mit expliziter Zustimmung, Nichtzustimmung
wird ohne Begründung akzeptiert und nicht hinterfragt.

\item Wir schaffen Sicherheit und diskutieren in einer entspannten Atmosphäre
miteinander.

\item Wir ermutigen rücksichtsvolles Fragen.

\item Wir wünschen uns couragiertes Verhalten.

\item Wir schützen unsere Grenzen und halten uns an die von anderen gesetzten
Grenzen. Du selbst bestimmt, was für dich grenzüberschreitend ist.
\end{itemize}

Die ZaPFika engagieren sich eine Atmosphäre von Sicherheit und Gewaltlosigkeit
für alle zu schaffen. Dies bedeutet insbesondere, dass:

\begin{itemize}
\item wir keine Form von sexualisierter Gewalt, Belästigung oder Diskriminierung
akzeptieren.

\item wir keine Beleidigungen, Trollen, Degradierungen oder persönliche Angriffe
akzeptieren.

\item wir nicht akzeptieren, wenn Menschen anstelle von Argumenten angegriffen
werden.
\end{itemize}


\section{Umsetzung%
  \label{umsetzung}%
}

Das Umsetzen dieses Verhaltenskodexes liegt in der Verantwortung aller
anwesender Personen; Teilnehmika, Helfika, Orga, etc. Wie in allem
zwischenmenschlichem Handeln, kommt es jedoch auch auf der ZaPF zu Problemen. Du
kannst dich mit Problemen jederzeit an die Vertrauenspersonen wenden. Im
Konfliktfall mit anderen Menschen oder bei persönlichen Problemen, die nur dich
betreffen, kannst du ihnen von deinem Problem erzählen. Falls du das wünschst,
können sie mit anderen Personen vermitteln, für dich bei der Tagungsleitung um
Hilfe bitten oder andere, z.B. externe, Hilfe organisieren. Du kannst mit so
vielen oder so wenigen Vertrauenspersonen sprechen, wie du möchtest.


\section{Geltungsbereich%
  \label{geltungsbereich}%
}

Dieser Verhaltenskodex gilt auf allen unmittelbar mit der ZaPF in Verbindung
stehenden Veranstaltungen. Im Speziellen gilt er auf allen ZaPFen und durch
Organe der ZaPF organisierten Veranstaltungen für alle anwesenden
Personen. Außerdem für alle die ZaPF repräsentierenden Personen während Sie ihre
Funktion ausüben. Mit der Anmeldung zu einer entsprechenden Veranstaltung,
spätestens jedoch mit dem Erscheinen, stimmst du zu, dich an diesen
Verhaltenskodex zu halten.


\section{Änderungshistorie%
  \label{anderungshistorie}%
}

Verabschiedet am 13. November 2022 auf der ZaPF in Hamburg.

\end{document}
